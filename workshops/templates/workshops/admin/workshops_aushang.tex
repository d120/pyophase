

% Mit Hilfe dieser LaTeX-Vorlage lassen sich die Aushänge für die Workshops generieren, auf denen sich die Teilnehmer eintragen können.
% Das generierte pdf enthält dann die Zettel für alle Workshops und braucht nur noch ausgedruckt zu werden.
% Soll am Seitenlayout nichts geändert werden, brauchen in der kopierten Vorlagendatei nur noch Informationen zwischen \begin{document} und \end{document} eingefügt zu werden.
% Pro Workshop wird ein \workshop Kommando mit 8 Parametern aufgerufen.
% Erklärung der Parameter:
% 1: Name des Workshops
% 2: Name des Workshop-Leiters
% 3: Art des Workshops (Vortrag, Workshop, Sport, ...)
% 4: Tag und Zeit
% 5: Ort
% 6: Beschreibungstext
% 7: Voraussetzungen (Benötigte Vorkenntnisse oder mitzubringende Dinge)
% 8: Max. Anzahl der Teilnehmer

\documentclass[a4paper]{article}

% die folgenden Pakete werden zur korrekten Darstellung von Umlauten benötigt
\usepackage[utf8]{inputenc}
\usepackage[ngerman]{babel}
\usepackage[T1]{fontenc}
% für URLs
\usepackage{url}
% die verwendete Schriftart
\usepackage{lmodern}
% folgendes wird für die Einschreibeliste benötigt
\usepackage{multicol}
\usepackage{forloop}
% ermöglicht die Fußzeile zu definieren
\usepackage{fancyhdr}
% Umlaute mit \enquote{in Anführungszeichen} setzen
\usepackage[autostyle=true,german=quotes,english=american]{csquotes}

% sämtliche Kopf- und Fußzeilenformatierung löschen und rechts die Seitennummer anzeigen
\pagestyle{fancy}
\fancyhf{}
\renewcommand{\headrulewidth}{0pt}
\fancyfoot[R]{\small Seite~\thepage}

% serifenlose Schriftart verwenden
\renewcommand{\familydefault}{\sfdefault}

% Workshop generieren
\newcommand{\workshop}[8]{
    % Seitennummer auf 1 zurückstellen und Workshoptitel links in Fußzeile anzeigen
    \setcounter{page}{1}
    \fancyfoot[L]{\small #1}
    % Kopfinformationen darstellen
    \begin{center}
        \huge #1
        \vspace{1em}\\
        \large Tutor: #2\\
        \large Art: #3\\
        \large Zeit: #4
        \hspace{2em}
        \large Ort: #5
    \end{center}
    #6
    \vspace{1em}\\
    Voraussetzungen: #7
    \vspace{1em}
    % Einschreibeliste einfügen und dann Seite beenden
    \signuplist{#8}
    \pagebreak
}

% generiert Einschreibeliste
\newcounter{i}
\newcommand{\signuplist}[1]{
    \begin{multicols}{2}
        \begin{enumerate}
            \forloop{i}{0}{\value{i} < #1}{
                \item \rule{0.4\textwidth}{0.4pt}
            }
        \end{enumerate}
    \end{multicols}
}

\begin{document}

% hier nun Workshops definieren



% --------  Workshop: {{ workshop.id }}  ----------


\workshop{{ '{' }}{{ workshop.title|tex_escape }}{{ '}' }} %Titel
{{ '{' }}{{ workshop.tutor_name|tex_escape }}{{ '}' }} %Tutor_Name
{{ '{' }}{{ workshop.workshop_type|tex_escape }}{{ '}' }} %Workshop_Type
{{ '{' }}{{ assignment.assigned_slot.date|date:'D'|tex_escape }} {{ assignment.assigned_slot.start_time|date:'H:i'|tex_escape }}{{ '}' }} %Assigned_Slot
{{ '{' }}{{ assignment.location|tex_escape }}{{ '}' }}  %Assigned_Location
{{ '{' }}{{ workshop.description|tex_escape }}{{ '}' }}  %Description
{{ '{' }}{{ workshop.participant_requirements|default:"keine"|tex_escape }}{{ '}' }}  %Participant_Requirements
{{ '{' }}{{ assignment.capacity|tex_escape }}{{ '}' }}  %Max_Participants





\end{document}

